\documentclass[11pt]{article}
\usepackage{geometry}
\geometry{letterpaper}

% TODO: roadmap

\usepackage{graphicx}
\usepackage{amssymb}
\usepackage{float}
\usepackage{tabularx}
\usepackage{multicol}
\usepackage{hyperref}
\hypersetup{
    colorlinks,
    citecolor=black,
    filecolor=blue,
    linkcolor=black,
    urlcolor=blue
}

\begin{document}


\begin{titlepage}
	\newcommand{\HRule}{\rule{\linewidth}{0.2mm}}
	\begin{center}
	\textsc{\LARGE McMaster University}\\[1.5cm]

	\textsc{\Large HydroSwarm}\\[0.5cm]
	\textsc{\large Software \& Mechatronics Capstone}\\[0.5cm]

	\HRule\\[0.4cm]
		{\huge\bfseries Hazard Analysis}\\[0.4cm]
	\HRule\\[0.4cm]

	\begin{minipage}[t][][t]{0.5\textwidth}
		\begin{flushleft} \large
			\emph{Authors:}\\
			Victor Velechovsky - \textit{001305263} \\
		\end{flushleft}
	\end{minipage}
	~
	\begin{minipage}[t][][t]{0.4\textwidth}
		\begin{flushright} \large
			\emph{Professor:} \\
			Dr. Alan Wassyng \\[0.4cm]
			\emph{Teaching Assistants:} \\
		\end{flushright}
	\end{minipage}\\[2cm]

	\includegraphics[width=0.3\textwidth]{logo.png} \\
	{\large Last compiled on \today}
	\end{center}

\end{titlepage}

\tableofcontents
\listoffigures

\vfill
\begin{figure}[H]
   \centering
   \noindent\begin{tabularx}{\textwidth}{| >{\centering\arraybackslash}m{0.2\textwidth} | >{\centering\arraybackslash}m{0.2\textwidth} | >{\centering\arraybackslash}m{0.2\textwidth} | >{\centering\arraybackslash}m{0.285\textwidth} |}
   \hline
   \textbf{Date} & \textbf{Revision} & \textbf{Comments} & \textbf{Author(s)} \\ \hline
   Dec 29, 2017 & 1.0 & Main content done for all sections & Christopher McDonald \\ \hline
   Jan 1, 2018 & 1.1 & Edited document & Sharon Platkin \\ \hline
   Jan 3, 2018 & 1.2 & Edited document & Jared Rayner \\ \hline
   Mar 4, 2018 & 1.3 & Revised  Figures& Christopher McDonald \\ \hline
   Mar 5, 2018 & 1.4 & Edited scope & Sharon Platkin \\ \hline
   \end{tabularx}
   \caption{Revision History}
\end{figure}
\newpage
\section{Introduction}

\subsection{Project Overview}
HydroSwarm is a swarm of autonomous robotic boats meant to carry out measurements over large bodies of water. For our project, these boats will be measuring water temperature, but the idea can be expanded to any number of other quantifiable measurements. Central to our work will be two major components. First, a small motorized boat, attached with a water temperature sensor, as well as a control unit that allows it to communicate with and be controlled by a centralized control unit. Second, a software package that can control a large group (\hyperref[sec:definitions]{swarm}) of these boats, with an algorithm focused on producing reliable, accurate, and fast measurements.\\

Our swarm will aim to cover large areas more quickly and cost-effectively than traditional products. To test the applicability of our project, we will demo it on a small scale body of water, such as a swimming pool, as well as develop a simulation to hypothetically prove the efficacy of the system on a larger scale.\\

Our project will be conducted between Fall 2018 –- Winter 2019 for our Engineering Capstone project at McMaster University, under the guidance of Dr. Alan Wassyng. We have four Software Engineering students, and one Mechatronics Engineering student. \\ \\

\subsection{Document Overview}

This document is intended to be a comprehensive guide to potential hazards that can occur during the
development or use of HydroSwarm. Although it attempts to be as comprehensive as possible, there
remains a possibility of other hazards emerging that we did not predict.

Each hazard will contain: a description, a fault tree, and a mitigation plan. This will all discussed
within the system boundary, which is defined in the section \textit{Scope & Boundary}.

\subsection{Naming Conventions and Terminology}
\label{sec:definitions}
The following terms and definitions will be used throughout this document:
\begin{itemize}
% Alphabetical order is highly preferred as it eases user navigation
\item \textbf{System}: The entire software and hardware package - including the boats,
boat hardware, control software, and server running the control software
\item \textbf{Swarm}: A large group of objects (in our case, motorized boats) that can communicate and perform acts as a group
\item \textbf{Insect}: A member of the swarm (in our case, a single motorized boat)
\item \textbf{Simulation}: The simulation will be used for demo purposes, mainly to show that
the system is valid with a large number of insects.
\item \textbf{Researcher}: A user that is interested in the data that is returned from the survey.
\item \textbf{System Administrator}: A user that controls the parameters of the \hyperref[sec:definitions]{\textbf{swarm}}.
\item \textbf{G.P.S.}: Global Positioning System
\item \textbf{Coverage area}: The area in which the insects are confined. This defines that space that the user wants to track and measure.
\item \textbf{API}: Application Programming Interface
\end{itemize}

\section{System}
\subsection{High-Level Design}

HydroSwarm contains both hardware components and software components, which can be categorized as
server components or insect components. Each insect will have an R.C. boat (equipped with a motor,
location tracking sensors, and measurement sensors), an RF communication device, and a micro-controller
that functions as the 'brain'.

The server will have a communication protocol, an algorithm to determine where to move each insect,
a data storage component to store measurement data, and a G.U.I. to interact with the user.

The design of the system is covered in detail in the \textit{System Design} document.

\subsection{Scope \& Boundary}
The hazards outlined below will cover the hardware, electrical and software aspects of the SmartServe system. \\ \\
Intentional misuse or destruction of the system will not be discussed as it is outside the scope of this document. The user is assumed to be using fully-functioning and maintained equipment including the table tennis balls, table and paddles. 

\section{Hazards}
The hazards outlined here can be categorized into 2 main categories: system and project. System hazards damage the user or the assets of the system where project hazards damage the integrity of the project and its success. The list of each can be found in Table \ref{table:hazard}. \\
\begin{table}[H]
\centering
\caption{Categorization of Hazards}
\label{table:hazard}
\begin{tabular}{ | >{\raggedright\arraybackslash}p{0.5\textwidth} | >{\raggedright\arraybackslash}p{0.5\textwidth} | }
\hline
\multicolumn{2}{|c|}{\textbf{Hazards}} \\ \hline
\textbf{System}   & \textbf{Project}   \\ \hline
\begin{itemize}
\item User is struck by an air-borne ball
\item User is caught in a Pinch Point
\item Electric Shock
\item User's appendage is caught in machine
\item Machine overheats
\item Machine's traversal is blocked
\end{itemize}
&
\begin{itemize}
\item CV frame rate is too low for accurate reading
\item Shooting Mechanism shoots inaccurately
\end{itemize}
\\ \hline
\end{tabular}
\end{table}


\subsection{Financial Losses}
\subsubsection*{Description}
Financial losses could occur either when an insect is damaged or lost, or when an insect causes
damage to the outside world. \\

A major source for insect damage would be water damage. Although the electrical components will be
housed inside the boat, adverse conditions could certainly be problematic.

If an insect collides with another insect, or with something in the external world, damage could be
caused to either the insect(s) or the external world.
\subsubsection*{Mitigation Plan}
\textit{Clear Coverage Area}: HydroSwarm should only be used in a body of water that has no
lifeforms or other fragile objects to minimize the risk of collisions. This includes, for example,
small branches floating in the water, which could cause problems for the movement of insects.

\textit{Double Check the Coverage Area}: It is important for the proper functionality of HydroSwarm
that the coverage area is defined accurately. 

\textit{Limit Movement Speed}: The maximum speed of the boats can be limited by limiting the voltage
sent to the motors, which could lessen the severity of collisions if they do happen. For development
purposes, velocities of the insects will be limited to 1 m/s.
\subsubsection*{Fault Tree}
\begin{figure}[H]
   \centering
   \includegraphics[width=0.7\textwidth]{diagram/fault-tree-financial-loss.png} % requires the graphicx package
   \caption{Fault Analysis Tree for \textit{User is struck by an air-borne ball}}
   \label{fig:ft-Air}
\end{figure}

\subsection{Bad Policy Decisions Made}
\subsubsection*{Description}
HydroSwarm's measurement data can be used by experts in various fields of study, to inform
policy decisions in the corporate, academic, or political spheres. This, of course, relies on
the assumption of accurate data, so an 'invisible' malfunction in HydroSwarm could lead to
malinformed decision making.
\subsubsection*{Mitigation Plan}
\textit{Cross reference results with a physical model.} In some situations it is possible to
approximate the results of HydroSwarm using a physical model. If there is a large discrepency
between the model and swarm, then further investigation is required to ensure the accuracy of
HydroSwarm.


\textit{Cross reference results with a second measurement source.} When available, cross referencing
HydroSwarm with another water temperature measuring device is extremely helpful in validating
the accuracy of the system.


\textit{Routinely test the system in known conditions.} If the user has available a body of water
with a known temperature map, the system should be routinely tested in it to validate the
accuracy of the measurement device. Note that both the temperature measurements and the locations
of these measurements should be validated, not simply the temperature measurements themselves.
\subsubsection*{Fault Tree}
\begin{figure}[H]
   \centering
   \includegraphics[width=0.7\textwidth]{img/ft-Air.png} % requires the graphicx package
   \caption{Fault Analysis Tree for \textit{User is struck by an air-borne ball}}
   \label{fig:ft-Air}
\end{figure}

\end{document}