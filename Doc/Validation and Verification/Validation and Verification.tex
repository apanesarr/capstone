\documentclass[11pt]{article}
\usepackage{geometry}
\geometry{letterpaper}

\usepackage{graphicx}
\usepackage{amssymb}
\usepackage{float}
\usepackage{tabularx}
\usepackage{multicol}
\usepackage{hyperref}
\hypersetup{
    colorlinks,
    citecolor=black,
    filecolor=blue,
    linkcolor=black,
    urlcolor=blue
}

\newcommand{\testcase}[8]{
\begin{center}
\begin{table}[H]
\begin{tabular}{|l r|}\hline&\\[-2mm]
	Test ID: {#1}	&Status: {#2} \\[-3mm]
	\multicolumn{2}{|c|}{\textbf{\large{{#3}}}}\\&\\\hline&\\[-3mm]
	\multicolumn{2}{|p{\textwidth}|}{Description: {#4}}\\[1mm]\hline&\\[-3mm]
	\multicolumn{2}{|p{\textwidth}|}{Pass/Fail Condition: {#5}}\\[1mm]\hline&\\[-3mm]
	\multicolumn{2}{|p{\textwidth}|}{Input: {#6}}\\[2mm]\hline
	\multicolumn{1}{|p{0.49\textwidth}}{Expected Results: {#7}}	&
    \multicolumn{1}{|p{0.45\textwidth}|}{Actual Results: {#8}}\\\hline&\\[-3mm]
\end{tabular}
\caption{User Interface Sign up Test}
\end{table}
\end{center}
}

\begin{document}

\begin{titlepage}
	\newcommand{\HRule}{\rule{\linewidth}{0.2mm}}
	\begin{center}
	\textsc{\LARGE McMaster University}\\[1.5cm]

	\textsc{\Large SmartServe}\\[0.5cm]
	\textsc{\large Software \& Mechatronics Capstone}\\[0.5cm]

	\HRule\\[0.4cm]
		{\huge\bfseries Verification and Validation}\\[0.4cm]
	\HRule\\[0.4cm]

	\begin{minipage}[t][][t]{0.5\textwidth}
		\begin{flushleft} \large
			\emph{Authors:}\\
			Victor Velechovsky - \textit{001305263} \\
		\end{flushleft}
	\end{minipage}
	~
	\begin{minipage}[t][][t]{0.4\textwidth}
		\begin{flushright} \large
			\emph{Professor:} \\
			Dr. Alan Wassyng \\[0.4cm]
			\emph{Teaching Assistants:} \\
			Bennett Mackenzie \\
			Nicholas Annable \\
			Stephen Wynn-Williams \\
			Viktor Smirnov
		\end{flushright}
	\end{minipage}\\[2cm]

	\includegraphics[width=0.3\textwidth]{logo.png} \\
	{\large Last compiled on \today}
	\end{center}

\end{titlepage}

\tableofcontents
\listoffigures
\listoftables

\vfill
\begin{figure}[H]
   \centering
   \noindent\begin{tabularx}{\textwidth}{| >{\centering\arraybackslash}m{0.2\textwidth} | >{\centering\arraybackslash}m{0.2\textwidth} | >{\centering\arraybackslash}m{0.2\textwidth} | >{\centering\arraybackslash}m{0.285\textwidth} |}
   \hline
   \textbf{Date} & \textbf{Revision} & \textbf{Comments} & \textbf{Author(s)} \\ \hline
   Feb 1, 2018 & 1.0 & Document structure and Headings & Christopher McDonald \\ \hline
   Feb 10, 2018 & 1.1 & Philosophy and Intro section & Christopher MCDonald \\ \hline
   Feb 13, 2018 & 1.2 & UI and DS Tests & Sharon Platkin \\ \hline
   Feb 15, 2018 & 1.3 & SM, SMC, SS, CV and SR & Christopher McDonald \& Harit Patel \& Sam Hamel \\ \hline
   Feb 16-17, 2018 & 1.4 & Executive Summary and Testing & Christopher McDonald \& Sharon Platkin  \\ \hline
   Apr 6, 2018 & 2.0 & Added integration tests, fixed grammar and added test instructions & Christopher McDonald  \\ \hline
   Apr 10, 2018 & 2.0 & Preformed tests and wrote executive summary & Christopher McDonald \& Sharon Platkin \& Sam Hamel  \\ \hline
   \end{tabularx}
   \caption{Revision History}
\end{figure}
\newpage
\section{Executive Summary of Testing}
% 63 tests, 7 fails
This testing phase completed on April 10th, 2018 and has yielded 56 passes and 7 failures. The failures occurred predominately in the Computer Vision and Shooting categories. The Computer Vision tests did not reach the desired accuracy levels and the shooting mechanism did not hit the desired zones during some of the tests. \\ \\
The team is confident in the coverage of the functional requirements based on the traceability matrices shown in Tables \ref{label1} and \ref{label2}. In these tables, every requirement has at least one test case which covers it. This is less true for non-functional requirements since 3 of the 18 have no test cases as shown in Tables \ref{label3} and \ref{label4}. These requirements were too difficult to test in an objective way and are thus omitted from this document.\\\\
These results accurately represent the priorities of the team to make the system more reliable, responsive and accurate. The team will continue developing the system until April 26th, after which the final presentation will take place on the 27th.
\section{Introduction}

\subsection{Project Overview}

\PROJECTNAME \space is a \hyperref[sec:definitions]{swarm} of autonomous robotic boats meant to carry out measurements over large bodies of water. For our project, these boats will be measuring water temperature, but the idea can be expanded to any number of other quantifiable measurements. Central to our work will be two major components. First, a small motorized boat, attached with a water temperature sensor, as well as a control unit that allows it to communicate with – and be controlled by – a centralized control unit. Second, a software package that can control a large group (\hyperref[sec:definitions]{swarm}) of these boats, with an algorithm focused on producing reliable, accurate, and fast measurements.\\

Our swarm will aim to cover large areas more quickly and cost-effectively than traditional products. To test the applicability of our project, we will demo it on a small scale body of water, such as a swimming pool, as well as develop a simulation to hypothetically prove the efficacy of the system on a larger scale.\\

Our project will be conducted between Fall 2018 – Winter 2019 for our Engineering Capstone project at McMaster University, under the guidance of Dr. Alan Wassyng. We have four Software Engineering students, and one Mechatronics Engineering student.

\subsection{Document Overview}
This document will provide details of all formal testing methods and results performed on the SmartServe system. The first part of testing includes detailing how it will be performed and the details of the system on which it is run. This matters due to details which can affect the testing outcomes like operating system, lighting or performance of hardware. The schedule for the test will also be detailed alongside the major deliverables to have clear outcomes to explain to stakeholders. The testing will be preformed off of the \textit{master} branch as it stands during the beginning of the testing phase. \\ \\
The actual testing will then be detailed as test cases based on what subsystem they are testing. As needed, the communication will be tested in between the subsystems to ensure communication is working as intended. Lastly, a test case-requirement matrix will be provided which maps what test cases test which requirement. This will give the reader a simple way to check if a requirement is fully satisfied. Supporting documents include the requirements which can be found \href{https://github.com/ChristopherMcDonald/SoftwareTronCapstone/blob/develop/documentation/Requirements/Requirements.pdf}{here}.

\subsection{Naming Conventions and Terminology}

\label{sec:definitions}
\begin{itemize}
\item \textbf{System}: The entire software and hardware package - including the boats,
boat hardware, control software, and server running the control software
\item \textbf{Swarm}: A large group of objects (in our case, motorized boats) that can communicate and perform acts as a group
\item \textbf{Insect}: A member of the swarm (in our case, a single motorized boat)
\item \textbf{Simulation}: The simulation will be used for demo purposes, mainly to show that
the system is valid with a large number of insects.
\item \textbf{Researcher}: A user that is interested in the data that is returned from the survey.
\item \textbf{System Administrator}: A user that controls the parameters of the \hyperref[sec:definitions]{\textbf{swarm}}.
\item \textbf{G.P.S.}: Global Positioning System
\end{itemize}

\section{Testing Philosophy}
\subsection{Approach}
Unit tests for Python modules will be written in PyUnit. All hardware units will be tested manually, according to the guidelines laid out in this document.

Tests will be performed upon completion of this document, and before the final presentation of April, 2019. All results will be included in the 'test cases' section of the document.

Each hardware test will be run three times, if possible, for a greater sense of confidence in the reliability of the product, though of course in a professional environment,
more thorough testing would be required.

\subsection{Schedule}
\begin{table}[H]
\centering
\label{my-label}
\begin{tabular}{|l|l|l|}
\hline
\textbf{Task} & \textbf{Date} & \textbf{Notes} \\ \hline
Write Test Cases & February 17, 2019 & N/A \\ \hline
Run Initial Test Plan & February 18, 2019 & NA \\ \hline
Fix Issues Discovered in Testing & February 18-19, 2019 & NA \\ \hline
Perform Second Round of Tests and Publish Final Results & February 20, 2019 & NA \\ \hline
\end{tabular}

\caption{Testing Schedule}
\end{table}
\subsection{Environment}
In order to support OpenCV object detection, the server will run on a high end laptop (2018 15'' Macbook pro) with Mac OS Mojave (10.14.3) and a quad-core Intel Core i7.

The object detection might require a certain environment to be used, in order to remove problematic images. Our Revision 0 testing was performed on the floor in Thode Library, though
for our final presentation we will likely chose a different location, such as one of the McMaster Basketball courts.

\subsection{Setup Instructions}
\begin{itemize}
\item Assemble insects
\item Attach a colored spherical identifier to each insect, each with a clearly distinct color (for our testing, we used 'Red' 'Green' and 'Blue' balls)
\item Upload the insect code to each insect arduino
\item Upload the master code to the master arduino
\item Check out the server code
\item Install OpenCV 3
\item Install all python dependencies
\item Turn on every arduino unit
\item Place the insects in their starting locations
\item Configure Parameters.py to match your setup (especially the SERIAL and COLORRANGE parameters)
\item Start the server code with $python Main.py$
\end{itemize}

\section{Test Cases}

% Test case macro has the following arguments:
% Test ID, Status, Name, Description, Pass/Fail Condition, Input, Expected Results, Actual Results

\subsection{Area Coverage Algorithm}

The area coverage algorithm is written in Python, so pyTest is used to write automated test
cases, wherever applicable.

\testcase
{ALG1}
{PASS}
{Run Performance #1}
{Simulation runtime performance with a 10x10 grid}
{Algorithm should run in simulation mode, on the test Macbook,
with a 100 x 100 grid, in under 10 seconds.}
{100x100 grid with 3 robots, started in random locations}
{Algorithm terminates within 10 seconds}
{Algorithm terminated within 10 seconds}

\testcase
{ALG2}
{PASS}
{Area Coverage #1}
{Algorithm covers every location in the coverage area}
{Every region in a 100x100 coverage area is covered}
{100x100 grid with 3 insects, started in random location}
{Every region is covered by the end of the simulation}
{Every region was covered by the end of the simulation}

\testcase
{ALG3}
{PASS}
{Start Locations #1}
{Algorithm works from any starting points}
{Algoritm covers area with randomly generated starting points}
{10x10 grid with (random) 1-5 robots placed at random locations}
{Every region is covered by the end of the simulation}
{Every region was covered by the end of the simulation}

\testcase
{ALG4}
{PASS}
{Arbitrary Number of Insects #1}
{Algorithm works with arbitrary insects}
{}
{10x10 grid with (random) 1-50 robots placed in the bottom left corner of the region}
{Every region is covered by the end of the simulation}
{Every region was covered by the end of the simulation}

\testcase
{ALG5}
{FAIL}
{Arbitrary Region Shapes}
{Algorithm works with arbitrarily shaped coverage areas}
{}
{Rectangle region, circular region, and octahedral region}
{Every region is covered by the end of the simulation}
{We don't currently support non-rectangular coverage areas}

\section{Test Case-Requirement Traceability Matrix}

\begin{table}[H]
\centering
\caption{Matrix to Match Tests to Functional Requirements [1]}
\label{label1}
\begin{tabular}{| l | l | l | l | l | l | l | l | l | l | l | l | l | l | l | l | l | l | l |}
\hline
\multicolumn{19}{|c|}{\textbf{Functional Requirement-Test Matrix}}                          \\ \hline
 & \tiny{F1} & \tiny{F2} & \tiny{F3} & \tiny{F4} & \tiny{F5} & \tiny{F6} & \tiny{F7} & \tiny{F8} & \tiny{F9} & \tiny{F10} & \tiny{F11} & \tiny{F12} & \tiny{F13} & \tiny{F14} & \tiny{F15} & \tiny{F16} & \tiny{F17} & \tiny{F18} \\ \hline
SMC1&X&X&X&X&X&&&&&&&&&&&&&X \\ \hline
SMC2&X&X&&&X&&&&&&&&&&&&&X \\ \hline
SMC3&X&&&X&&&&&&&&&&&&&& \\ \hline
SMC4&X&&&&&&&&&&&&&&&&& \\ \hline
SMC5&X&&X&&&&&&&&&&&&&&& \\ \hline
SMC6&&&&&&&&&&&&&&&&&& \\ \hline
CV1&&&&&X&&&&&&&&&&&&&X \\ \hline
CV2&&&&&X&&&&&&&&&&&&& \\ \hline
CV3&&&&&X&&&&&&&&&&&&& \\ \hline
CV4&&&&&X&&&&&&&&&&&&&X \\ \hline
CV5&&&&&X&&&&&&&&&&&&&X \\ \hline
CV6&&&&&&X&&&&&&&&&&&&X \\ \hline
CV7&&&&&X&&&&&&&&&&&&& \\ \hline
CV8&&&&&X&&&&&&&&&&&&& \\ \hline
CV9&&&&&X&&&&&&&&&&&&& \\ \hline
CV10&&&&&&X&&&&&&&&&&&& \\ \hline
CV11&&&&&X&&&&&&&&&&&&& \\ \hline
CV12&&&&&X&&&&&&&&&&&&& \\ \hline
SR1&&&&&&&&&&&&&&X&X&&& \\ \hline
SR2&&&&&&&X&&&&&&&X&X&&& \\ \hline
SR3&&&&&&&&&&&&&&&&&& \\ \hline
SR4&&&&&&&&&&&&&&X&&&& \\ \hline
SR5&&&&&&&&&&&&&&X&&&& \\ \hline
SM1&X&&X&&X&&&&&&&&&&&&& \\ \hline
SM2&X&X&&&X&&&&&&&&&&&&&X \\ \hline
SM3&&&&&&&&&&&&&&&&&& \\ \hline
DS1&&&&&&&&X&&&&&&&&&& \\ \hline
DS2&X&X&X&X&X&X&&&&&&&&&&&&X \\ \hline
DS3&&&&&&X&&&&&&&&&&&& \\ \hline
DS4&&&&&&&&&X&&&&&&&&& \\ \hline
DS5&&&&&X&X&X&X&&&&&X&&&&& \\ \hline
\end{tabular}
\end{table}

\begin{table}[H]
\centering
\caption{Matrix to Match Tests to Functional Requirements [2]}
\label{label2}
\begin{tabular}{| l | l | l | l | l | l | l | l | l | l | l | l | l | l | l | l | l | l | l |}
\hline
\multicolumn{19}{|c|}{\textbf{Functional Requirement-Test Matrix}}                          \\ \hline
 & \tiny{F1} & \tiny{F2} & \tiny{F3} & \tiny{F4} & \tiny{F5} & \tiny{F6} & \tiny{F7} & \tiny{F8} & \tiny{F9} & \tiny{F10} & \tiny{F11} & \tiny{F12} & \tiny{F13} & \tiny{F14} & \tiny{F15} & \tiny{F16} & \tiny{F17} & \tiny{F18} \\ \hline
UI1&&&&&&&&X&X&X&X&X&X&&&X&X& \\ \hline
UI2&&&&&&&&X&X&X&X&X&X&&&X&X& \\ \hline
UI3&&&&&&&&&&&&&&&&X&& \\ \hline
UI4&&&&&&&&X&&&&&&&&&& \\ \hline
UI5&&&&&&&&X&X&&&&&&&&& \\ \hline
UI6&&&&&X&X&X&&&&&&X&X&&&& \\ \hline
SS1&&&&&&&X&&&&&&&X&X&&& \\ \hline
SS2&&&&&&&X&&&&&&&X&X&&& \\ \hline
SS3&&&&&&&&&&&&&&&&&& \\ \hline
SS4&&&&&&&&&&&&&&X&&&& \\ \hline
SS5&&&&&&&&&&&&&&&X&&& \\ \hline
SS6&&&&&&&&&&&&&&X&&&& \\ \hline
SS7&&&&&&&&&&&&&&X&&&& \\ \hline
SS8&&&&&X&X&&&&&&&&&&&&X \\ \hline
SS9&&&&&X&X&&&&&&&&&&&&X \\ \hline
SS10&&&&&X&X&&&&&&&&&&&&X \\ \hline
SS11&&&&&X&X&&&&&&&&&&&&X \\ \hline
SS12&&&&&&X&&X&X&&&&X&&&&&X \\ \hline
SS13&&&&&&X&&X&X&&&&X&&&&&X \\ \hline
SS14&&&&&&&&X&&&&&&&&&& \\ \hline
SS15&&&&&&X&&&&&&&&&&&& \\ \hline
SS16&&&&&&&&&X&&&&&&&&& \\ \hline
SS17&X&X&X&X&X&&&&&&&&&&&&&X \\ \hline
SS18&X&X&X&X&X&&&&&&&&&&&&&X \\ \hline
SS19&X&&X&&&&&&&&&&&&&&& \\ \hline
SS20&X&&X&&&&&&&&&&&&&&& \\ \hline
SS21&X&X&&&X&&&&&&&&&&&&&X \\ \hline
SS22&X&X&&&X&&&&&&&&&&&&&X \\ \hline
HI1&X&X&X&X&&&&&&&&&&&X&&& \\ \hline
HI2&X&X&X&X&&&&&&&&&&&X&&& \\ \hline
HI3&X&X&X&X&&&&&&&&&&&X&&& \\ \hline
HI4&X&X&X&X&&&&&&&&&&&X&&& \\ \hline
\end{tabular}
\end{table}


\begin{table}[H]
\centering
\caption{Matrix to Match Tests to Non-Functional Requirements [1]}
\label{label3}
\begin{tabular}{| l | l | l | l | l | l | l | l | l | l | l | l | l | l | l | l | l | l | l |}
\hline
\multicolumn{19}{|c|}{\textbf{Non-Functional Requirement-Test Matrix}}                          \\ \hline
 & \tiny{LF1} & \tiny{UH1} & \tiny{UH2} & \tiny{P1} & \tiny{P2} & \tiny{P4} & \tiny{P5} & \tiny{OE2} & \tiny{MS2} & \tiny{S1} & \tiny{S2} & \tiny{P1} & \tiny{LC1} & \tiny{HS1} & \tiny{HS2} & \tiny{HS3} & \tiny{HS4} & \tiny{HS5} \\ \hline
SMC1&&&&&&&&&&&&&&X&X&&& \\ \hline
SMC2&&&&&&&&&&&&&&X&X&&& \\ \hline
SMC3&&&&&&&&&&&&&&X&&&& \\ \hline
SMC4&&&&&&&&&&&&&&X&&&& \\ \hline
SMC5&&&&&&&&&&&&&&X&&&& \\ \hline
SMC6&&&&&&&&&&&&&&&&&&X \\ \hline
CV1&&&&&&&&X&&&&&&&&&& \\ \hline
CV2&&&&&&&&&&&&&&&&&& \\ \hline
CV3&&&&&&&&&&&&&&&&&& \\ \hline
CV4&&&&&&&&&&&&&&&&&& \\ \hline
CV5&&&&&&&&&&&&&&&&&& \\ \hline
CV6&&&&&&&&&&&&&&&&&& \\ \hline
CV7&&&&&&&&&&&&&&&&&& \\ \hline
CV8&&&&&&&&&&&&&&&&&& \\ \hline
CV9&&&&&&&&&&&&&&&&&& \\ \hline
CV10&&&&&&&&&&&&&&&&&& \\ \hline
CV11&&&&&&&&&&&&&&&&&& \\ \hline
CV12&&&&&&&&&&&&&&&&&& \\ \hline
SR1&&&&&&&&&&&&&&X&X&&& \\ \hline
SR2&&&&&&&&&&&&&&X&&&& \\ \hline
SR3&&&&&&&&&&&&&&X&&&& \\ \hline
SR4&&&&&&&&&&&&&&&&&& \\ \hline
SR5&&&&&&&&&&&&&&&&&& \\ \hline
SM1&&&&&&&&&&&&&&X&&&& \\ \hline
SM2&&&&&&&&&&&&&&X&X&&& \\ \hline
SM3&&&&&&&&&&&&&&X&&&& \\ \hline
DS1&&&&&&X&&&&X&&&&&&&& \\ \hline
DS2&&&&&&&&&&&&&&X&X&&& \\ \hline
DS3&&&&&X&&&&&&&&&&&&& \\ \hline
DS4&&&&&&&&&&&&&&&&&& \\ \hline
DS5&&&&&&&&&X&&X&&&&&&& \\ \hline
\end{tabular}
\end{table}

\begin{table}[H]
\centering
\caption{Matrix to Match Tests to Non-Functional Requirements [2]}
\label{label4}
\begin{tabular}{| l | l | l | l | l | l | l | l | l | l | l | l | l | l | l | l | l | l | l |}
\hline
\multicolumn{19}{|c|}{\textbf{Non-Functional Requirement-Test Matrix}}                          \\ \hline
 & \tiny{LF1} & \tiny{UH1} & \tiny{UH2} & \tiny{P1} & \tiny{P2} & \tiny{P4} & \tiny{P5} & \tiny{OE2} & \tiny{MS2} & \tiny{S1} & \tiny{S2} & \tiny{P1} & \tiny{LC1} & \tiny{HS1} & \tiny{HS2} & \tiny{HS3} & \tiny{HS4} & \tiny{HS5} \\ \hline
UI1&X&X&X&&X&X&X&&X&X&&X&&&&&& \\ \hline
UI2&X&X&X&X&&&&&&&&&&&&&& \\ \hline
UI3&&&X&X&&&&&&&&&&&&&& \\ \hline
UI4&&&X&&&X&&&&X&&&&&&&& \\ \hline
UI5&X&X&X&X&&&&&&&&&&&&&& \\ \hline
UI6&X&X&X&X&&&&&X&&&&&&&&& \\ \hline
SS1&&&&&&&&&&&&&&&X&&& \\ \hline
SS2&&&&&&&&&&&&&&&X&&& \\ \hline
SS3&&&&&&&&&&&&&&&X&&& \\ \hline
SS4&&&&&&&&&&&&&&&X&&& \\ \hline
SS5&&&&&&&&&&&&&&&X&&& \\ \hline
SS6&&&&&&&&&&&&&&&&&& \\ \hline
SS7&&&&&&&&&&&&&&&&&& \\ \hline
SS8&&&&&&&&&&&&&&&&&& \\ \hline
SS9&&&&&&&&&&&&&&&&&& \\ \hline
SS10&&&&&&&&&&&&&&&&&& \\ \hline
SS11&&&&&&&&&&&&&&&&&& \\ \hline
SS12&&&&&X&X&&&&X&&&&&&&& \\ \hline
SS13&&&&&X&X&&&&X&&&&X&X&&& \\ \hline
SS14&&&&&&X&&&&X&&&&X&X&&& \\ \hline
SS15&&&&&X&&&&&&&&&&&&& \\ \hline
SS16&&&&&&&&&&&&&&&&&& \\ \hline
SS17&&&&&&&&&&&&&&X&X&&& \\ \hline
SS18&&&&&&&&&&&&&&X&X&&& \\ \hline
SS19&&&&&&&&&&&&&&X&&&& \\ \hline
SS20&&&&&&&&&&&&&&X&&&& \\ \hline
SS21&&&&&&&&&&&&&&X&X&&& \\ \hline
SS22&&&&&&&&&&&&&&X&X&&& \\ \hline
HI1&&&&&&&&&&&&&&X&X&&& \\ \hline
HI2&&&&&&&&&&&&&&X&X&&& \\ \hline
HI3&&&&&&&&&&&&&&X&X&&& \\ \hline
HI4&&&&&&&&&&&&&&X&X&&& \\ \hline
\end{tabular}
\end{table}

\section{Appendix}
\begin{figure}[H]
\begin{center}
\begin{tabular}{ | m{2cm} | m{2cm}| m{3cm} |m{3cm} | } 
 \hline
   \textbf{Shot ID} & \textbf{Zone ID} & \textbf{Roll (degrees, x-axis)} & \textbf{Pitch (degrees, y-axis )} \\ \hline
   48 & 17 & 0 & 20  \\ \hline 
   91 & 12 & 90 & 10  \\ \hline 
   148 & 5 & 180 & 10 \\ \hline 
   193 & 2 & 270 & 0 \\ \hline 
\end{tabular}
   \caption{Shot details for tested shots}
\end{center}
\end{figure}

\end{document}

% Test Case Template
\begin{center}
\begin{table}[H]
\begin{tabular}{|l r|}\hline&\\[-2mm]
	Test ID: XXY	&Status: ????\\[-3mm]
	\multicolumn{2}{|c|}{\textbf{\large{Title???}}}\\&\\\hline&\\[-3mm]
	\multicolumn{2}{|p{\textwidth}|}{Description: ????}\\[1mm]\hline&\\[-3mm]
	\multicolumn{2}{|p{\textwidth}|}{Pass/Fail Condition: ????}\\[1mm]\hline&\\[-3mm]
	\multicolumn{2}{|p{\textwidth}|}{Pre-Conditions: ????}\\[4mm]
	\multicolumn{2}{|p{\textwidth}|}{Input: ????}\\[2mm]\hline
	\multicolumn{1}{|p{0.49\textwidth}}{Expected Results: ????}	&\multicolumn{1}{|p{0.45\textwidth}|}{Actual Results: ????}\\\hline&\\[-3mm]
	\multicolumn{2}{|p{\textwidth}|}{Post-Conditions: ????}\\\hline
\end{tabular}
\caption{Title???}
\end{table}
\end{center}
