\documentclass[11pt]{article}
\usepackage{geometry}
\geometry{letterpaper}

\usepackage{graphicx}
\usepackage{amssymb}
\usepackage{float}
\usepackage{tabularx}
\usepackage{multicol}
\usepackage{hyperref}
\hypersetup{
    colorlinks,
    citecolor=black,
    filecolor=blue,
    linkcolor=black,
    urlcolor=blue
}

\newcommand{\testcase}[8]{
\begin{center}
\begin{table}[H]
\begin{tabular}{|l r|}\hline&\\[-2mm]
	Test ID: {#1}	&Status: {#2} \\[-3mm]
	\multicolumn{2}{|c|}{\textbf{\large{{#3}}}}\\&\\\hline&\\[-3mm]
	\multicolumn{2}{|p{\textwidth}|}{Description: {#4}}\\[1mm]\hline&\\[-3mm]
	\multicolumn{2}{|p{\textwidth}|}{Pass/Fail Condition: {#5}}\\[1mm]\hline&\\[-3mm]
	\multicolumn{2}{|p{\textwidth}|}{Input: {#6}}\\[2mm]\hline
	\multicolumn{1}{|p{0.49\textwidth}}{Expected Results: {#7}}	&
    \multicolumn{1}{|p{0.45\textwidth}|}{Actual Results: {#8}}\\\hline&\\[-3mm]
\end{tabular}
\caption{User Interface Sign up Test}
\end{table}
\end{center}
}

\begin{document}

\begin{titlepage}
	\newcommand{\HRule}{\rule{\linewidth}{0.2mm}}
	\begin{center}
	\textsc{\LARGE McMaster University}\\[1.5cm]

	\textsc{\Large SmartServe}\\[0.5cm]
	\textsc{\large Software \& Mechatronics Capstone}\\[0.5cm]

	\HRule\\[0.4cm]
		{\huge\bfseries Verification and Validation}\\[0.4cm]
	\HRule\\[0.4cm]

	\begin{minipage}[t][][t]{0.5\textwidth}
		\begin{flushleft} \large
			\emph{Authors:}\\
			Victor Velechovsky - \textit{001305263} \\
		\end{flushleft}
	\end{minipage}
	~
	\begin{minipage}[t][][t]{0.4\textwidth}
		\begin{flushright} \large
			\emph{Professor:} \\
			Dr. Alan Wassyng \\[0.4cm]
			\emph{Teaching Assistants:} \\
		\end{flushright}
	\end{minipage}\\[2cm]

	\includegraphics[width=0.3\textwidth]{logo.png} \\
	{\large Last compiled on \today}
	\end{center}

\end{titlepage}

\tableofcontents

% TODO remove these?
\listoffigures
\listoftables

\vfill
\begin{figure}[H]
   \centering
   \noindent\begin{tabularx}{\textwidth}{| >{\centering\arraybackslash}m{0.2\textwidth} | >{\centering\arraybackslash}m{0.2\textwidth} | >{\centering\arraybackslash}m{0.2\textwidth} | >{\centering\arraybackslash}m{0.285\textwidth} |}
   \hline
   \textbf{Date} & \textbf{Revision} & \textbf{Comments} & \textbf{Author(s)} \\ \hline
   Feb 1, 2019 & 1.0 & Document template added & All authors \\ \hline
   Feb 5, 2019 & 1.0 & Rough outline of test cases added & All authors \\ \hline
   Feb 15, 2019 & 1.0 & All test cases added, other sections as well & All authors \\ \hline
   Feb 18, 2019 & 1.0 & Test cases run, results added & All authors \\ \hline
   Feb 19, 2019 & 1.0 & Traceability matrix added & All authors \\ \hline
   \end{tabularx}
   \caption{Revision History}
\end{figure}
\newpage
\section{Executive Summary of Testing}
% TODO
\section{Introduction}

\subsection{Project Overview}

\PROJECTNAME \space is a \hyperref[sec:definitions]{swarm} of autonomous robotic boats meant to carry out measurements over large bodies of water. For our project, these boats will be measuring water temperature, but the idea can be expanded to any number of other quantifiable measurements. Central to our work will be two major components. First, a small motorized boat, attached with a water temperature sensor, as well as a control unit that allows it to communicate with – and be controlled by – a centralized control unit. Second, a software package that can control a large group (\hyperref[sec:definitions]{swarm}) of these boats, with an algorithm focused on producing reliable, accurate, and fast measurements.\\

Our swarm will aim to cover large areas more quickly and cost-effectively than traditional products. To test the applicability of our project, we will demo it on a small scale body of water, such as a swimming pool, as well as develop a simulation to hypothetically prove the efficacy of the system on a larger scale.\\

Our project will be conducted between Fall 2018 – Winter 2019 for our Engineering Capstone project at McMaster University, under the guidance of Dr. Alan Wassyng. We have four Software Engineering students, and one Mechatronics Engineering student.

\subsection{Document Overview}

\subsection{Naming Conventions and Terminology}

\label{sec:definitions}
\begin{itemize}
\item \textbf{System}: The entire software and hardware package - including the boats,
boat hardware, control software, and server running the control software
\item \textbf{Swarm}: A large group of objects (in our case, motorized boats) that can communicate and perform acts as a group
\item \textbf{Insect}: A member of the swarm (in our case, a single motorized boat)
\item \textbf{Simulation}: The simulation will be used for demo purposes, mainly to show that
the system is valid with a large number of insects.
\item \textbf{Researcher}: A user that is interested in the data that is returned from the survey.
\item \textbf{System Administrator}: A user that controls the parameters of the \hyperref[sec:definitions]{\textbf{swarm}}.
\item \textbf{G.P.S.}: Global Positioning System
\end{itemize}

\section{Testing Philosophy}
\subsection{Approach}
Unit tests for Python modules will be written in PyUnit. All hardware units will be tested manually, according to the guidelines laid out in this document.

Tests will be performed upon completion of this document, and before the final presentation of April, 2019. All results will be included in the 'test cases' section of the document.

Each hardware test will be run three times with slightly different inputs, when possible, for a greater sense of confidence in the reliability of the product, though of course in a professional environment,
more thorough testing would be required.

\subsection{Schedule}
\begin{table}[H]
\centering
\label{my-label}
\begin{tabular}{|l|l|l|}
\hline
\textbf{Task} & \textbf{Date} & \textbf{Notes} \\ \hline
Write Test Cases & February 17, 2019 & N/A \\ \hline
Run Initial Test Plan & February 18, 2019 & NA \\ \hline
Fix Issues Discovered in Testing & February 18-19, 2019 & NA \\ \hline
Perform Second Round of Tests and Publish Final Results & February 20, 2019 & NA \\ \hline
\end{tabular}

\caption{Testing Schedule}
\end{table}
\subsection{Environment}
In order to support OpenCV object detection, the server will run on a high end laptop (2018 15'' Macbook pro) with Mac OS Mojave (10.14.3) and a quad-core Intel Core i7.

The object detection might require a certain environment to be used, in order to remove problematic images. Our Revision 0 testing was performed on the floor in Thode Library, though
for our final presentation we will likely chose a different location, such as one of the McMaster Basketball courts.

\subsection{Setup Instructions}
\begin{itemize}
\item Assemble insects
\item Attach a colored spherical identifier to each insect, each with a clearly distinct color (for our testing, we used 'Red' 'Green' and 'Blue' balls)
\item Upload the insect code to each insect arduino
\item Upload the master code to the master arduino
\item Check out the server code
\item Install OpenCV 3
\item Install all python dependencies
\item Turn on every arduino unit
\item Place the insects in their starting locations
\item Configure Parameters.py to match your setup (especially the SERIAL and COLORRANGE parameters)
\item Start the server code with $python Main.py$
\end{itemize}

\section{Test Cases}

% Test case macro has the following arguments:
% Test ID, Status, Name, Description, Pass/Fail Condition, Input, Expected Results, Actual Results

\subsection{Area Coverage Algorithm}

The area coverage algorithm is written in Python, so pyTest is used to write automated test
cases, wherever applicable.

\subsection{Integration Tests}

% TODO - for stuff that tests multiple subsystems at once, there's a few requirements that need this like NF11, NF10

\testcase
{ALG1}
{PASS}
{Run Performance}
{Simulation runtime performance with a 10x10 grid}
{Algorithm should run in simulation mode, on the test Macbook,
with a 100 x 100 grid, in under 10 seconds.}
{100x100 grid with 3 robots, started in random locations}
{Algorithm terminates within 10 seconds}
{Algorithm terminated within 10 seconds}

\testcase
{ALG2}
{PASS}
{Area Coverage}
{Algorithm covers every location in the coverage area}
{Every region in a 100x100 coverage area is covered}
{100x100 grid with 3 insects, started in random location}
{Every region is covered by the end of the simulation}
{Every region was covered by the end of the simulation}

\testcase
{ALG3}
{PASS}
{Start Locations}
{Algorithm works from any starting points}
{Algoritm covers area with randomly generated starting points}
{10x10 grid with (random) 1-5 robots placed at random locations}
{Every region is covered by the end of the simulation}
{Every region was covered by the end of the simulation}

\testcase
{ALG4}
{PASS}
{Arbitrary Number of Insects}
{Algorithm works with arbitrary insects}
{}
{10x10 grid with (random) 1-50 robots placed in the bottom left corner of the region}
{Every region is covered by the end of the simulation}
{Every region was covered by the end of the simulation}

\testcase
{ALG5}
{FAIL}
{Arbitrary Region Shapes}
{Algorithm works with arbitrarily shaped coverage areas}
{}
{Rectangle region, circular region, and octahedral region}
{Every region is covered by the end of the simulation}
{We don't currently support non-rectangular coverage areas}

\testcase
{ALG6}
{PASS}
{No Repeated Measurements}
{Algorithm does not repeat measurements}
{}
{Random coverage area, random number of robots, random start locations }
{Every region is measured only once}
{Every region was measures only once}

\testcase
{ALG7}
{FAIL}
{Return to starting position}
{The insects will return to the start position when measurements are complete}
{}
{Random coverage area, random number of robots, random start locations}
{Insects return to their original positions}
{Feature has not been implemented yet}

\testcase
{ALG8}
{PASS}
{No leaked data}
{The algorithm will not write any user data to public file resources on the server}
{Leaked data is a privacy concern}
{Random simulation run}
{Program does not write to public files on the machine}
{There is no file IO}

\section{Test Case-Requirement Traceability Matrix}

% TODO idk why the table looks messed up

\begin{table}[H]
\centering
\caption{Functional Requirements [1]}
\label{label1}
\begin{tabular}{| l | l | l | l | l | l | l | l | l | l | 1 |}
\hline
\multicolumn{10}{|c|}{\textbf{Functional Requirement-Test Matrix}} \\ \hline
 & \tiny{F1} & \tiny{F2} & \tiny{F3} & \tiny{F4} & \tiny{F5} & \tiny{F6} & \tiny{F7} & \tiny{F8} & \tiny{F9} & \tiny{F10} \\ \hline
ALG1&&&&&&&&&& \\ \hline
ALG2&&&&&&&&&& \\ \hline
ALG3&&&&&&&&&& \\ \hline
ALG4&&&&&&&&&& \\ \hline
ALG5&&&&&&&&&& \\ \hline
ALG6&&&&&&X&&&& \\ \hline
ALG7&&&&&&&X&&& \\ \hline
ALG8&&&&&&&&&& \\ \hline
\end{tabular}
\end{table}

\begin{table}[H]
\centering
\caption{Non Functional Requirements [1]}
\label{label1}
\begin{tabular}{| l | l | l | l | l | l | l | l | l | l | 1 | 1 | 1 | 1 |}
\hline
\multicolumn{10}{|c|}{\textbf{Functional Requirement-Test Matrix}} \\ \hline
 & \tiny{NF1} & \tiny{NF2} & \tiny{NF3} & \tiny{NF4} & \tiny{NF5} & \tiny{NF6} & \tiny{NF7} & \tiny{NF8} & \tiny{NF9} & \tiny{NF10}
 & \tiny{NF11} & \tiny{NF12} & \tiny{NF13} \\ \hline
ALG1&&&X&&&&&&&&&& \\ \hline
ALG2&&&&&&&&&&&&& \\ \hline
ALG3&&&&&&&&&&&&& \\ \hline
ALG4&&&&&X&&X&&&&&& \\ \hline
ALG5&&&&&&&&&&&&& \\ \hline
ALG6&&&&&&&&&&&&& \\ \hline
ALG7&&&&&&&&&&&&& \\ \hline
ALG8&&&&&&&&&X&&&& \\ \hline
\end{tabular}
\end{table}

\section{Discussion of Results}

The results of the test cases were expected. All tests that failed were a result of
features that have not been implemented yet. All other tests passed, making us confident that
project development is going on track.

\section{Appendix}
% TODO - idk why removing the appendix breaks the build

\end{document}
